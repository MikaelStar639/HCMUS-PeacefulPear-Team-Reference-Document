\documentclass[11pt,a4paper]{article}
\usepackage[utf8]{inputenc}
\usepackage[T1]{fontenc}
\usepackage[margin=0.6in]{geometry}
\usepackage{multicol}
\usepackage{listings}
\usepackage{xcolor}
\usepackage{titlesec}
\usepackage{fancyhdr}

% ===== Header / Footer =====
\pagestyle{fancy}
\fancyhf{}
\fancyhead[L]{\small University of Science - VNUHCM \quad | Team: HCMUS-PeacefulPear}
\fancyhead[R]{\thepage}
\setlength{\headheight}{15pt}

% ===== Code style =====
\lstset{
  basicstyle=\ttfamily\footnotesize,
  keywordstyle=\color{blue}\bfseries,
  commentstyle=\color{gray},
  stringstyle=\color{red},
  breaklines=true,
  columns=fullflexible,
  frame=single,
  showstringspaces=false,
  tabsize=2,
  language=C++,
  morekeywords={ll,ull,__int128}
}

% ===== Section formatting =====
\titleformat{\section}{\bfseries}{}{0pt}{}
\titleformat{\subsection}{\normalsize\bfseries}{}{0pt}{}
\titlespacing*{\section}{0pt}{6pt plus 2pt}{2pt}
\titlespacing*{\subsection}{0pt}{4pt plus 1pt}{2pt}

\begin{document}

% ---------- Table of Contents ----------
\setcounter{tocdepth}{2}
\tableofcontents
\newpage

% ---------- Main Content ----------
\begin{multicols*}{2}

\section{Data Structures}

\subsection{Sparse Table}
\begin{lstlisting}
int a[N], st[LG + 1][N];
void preprocess() {
    for (int i = 1; i <= n; ++i) st[0][i] = a[i];
    for (int j = 1; j <= LG; ++j)
        for (int i = 1; i + (1 << j) - 1 <= n; ++i)
            st[j][i] = min(st[j - 1][i], st[j - 1][i + (1 << (j - 1))]);
}

int query(int l, int r) {
    int k = __lg(r - l + 1);
    return min(st[k][l], st[k][r - (1 << k) + 1]);
}

//query sum:
int querySum(int l, int r) {
    int len = r - l + 1;
    int sum = 0;
    for (int j = 0; (1 << j) <= len; ++j)
        if (len >> j & 1) {
            sum = sum + st[j][l];
            l = l + (1 << j);
        }
    return sum;
}
\end{lstlisting}

\subsection{Fenwick Tree}
\begin{lstlisting}
void update(int i, int val){
    for (; i <= n; i += i & -i) bit[i] += val;
}

int get(int i){
    int res = 0;
    for (; i; i -= i & -i) res += bit[i];
    return res;
}
\end{lstlisting}

\subsection{Segment Tree}
\begin{lstlisting}
struct Segment_tree{
	int st[4 * N], lazy[4 * N];
	
    void apply(int id, int c){
        update(st[id], c);
        update(lazy[id], c);
    }

	void down(int id, int l, int r){
		int c = lazy[id]; lazy[id] = 0;
        apply(id << 1, c); apply (id << 1 | 1, c);
	}
	
	void build(int id, int l, int r){
		if (l == r){
			st[id] = a[l];
			return;
		}
		
		int mid = (l + r) >> 1;
		build(id << 1, l, mid);
		build(id << 1 | 1, mid + 1, r);
		
		st[id] = merge(st[id << 1], st[id << 1 | 1]);
	}
	
	void update(int id, int l, int r, int u, int v, int x){
		if (r < u || v < l) return;
		if (u <= l && r <= v){
			apply(id, x);
			return;
		}
		
		down(id, l, r);
		
		int mid = (l + r) >> 1;
		update(id << 1, l, mid, u, v, x);
		update(id << 1 | 1, mid + 1, r, u, v, x);
		
		st[id] = merge(st[id << 1], st[id << 1 | 1]);
	}
	
	int get(int id, int l, int r, int u, int v){
		if (r < u || v < l) return -INF;
		if (u <= l && r <= v) return st[id];
		
		down(id, l, r);
		
		int mid = (l + r) >> 1;
		return merge(get(id << 1, l, mid, u, v), get(id << 1 | 1, mid + 1, r, u, v));
	}
} ST;
\end{lstlisting}

\subsection{Sigma Tree}
\begin{lstlisting}
struct Sigma_Tree{
    int st[2 * N];
 
    void init(){
        For(i, 1, n) st[i + n - 1] = a[i];
        ForD(i, n - 1, 1) st[i] = merge(st[i << 1], st[i << 1 | 1]); 
    }
 
 	void update(int p, int val){
 		p += n - 1;
 		st[p] = val;
 		for (; p > 1; p >>= 1) st[p >> 1] = merge(st[p], st[p ^ 1]);
 	}
 	
    int get(int l, int r){
        int res = 0;
        for (l += n - 1, r += n - 1; l <= r; l >>= 1, r >>= 1){
            if (l & 1) res = merge(res, st[l++]);
            if (!(r & 1)) res = merge(res, st[r--]);
        }
        return res;
    }
} ST;
\end{lstlisting}

\subsection{Persistent Segment Tree}
\begin{lstlisting}
struct Node {
    int left, right;    // ID of left child & right child
    long long ln;       // Max value of node
    Node() {}
    Node(long long ln, int left, int right) : ln(ln), left(left), right(right) {}
} it[N];         // Each node has a position in this array, called ID
int nNode;

int ver[N];            // ID of root in each version

// Update max value of a node
inline void refine(int cur) {
    it[cur].ln = max(it[it[cur].left].ln, it[it[cur].right].ln);
}

// Update a range, and return new ID of node
int update(int l, int r, int u, int x, int oldId) {
    if (l == r) {
        ++nNode;
        it[nNode] = Node(x, 0, 0);
        return nNode;
    }

    int mid = (l + r) >> 1;
    int cur = ++nNode;

    if (u <= mid) {
        it[cur].left = update(l, mid, u, x, it[oldId].left);
        it[cur].right = it[oldId].right;
        refine(cur);
    }
    else {
        it[cur].left = it[oldId].left;
        it[cur].right = update(mid+1, r, u, x, it[oldId].right);
        refine(cur);
    }

    return cur;
}

// Get max of range. Same as usual IT
int get(int nodeId, int l, int r, int u, int v) {
    if (v < l || r < u) return -1;
    if (u <= l && r <= v) return it[nodeId].ln;

    int mid = (l + r) >> 1;
    return max(get(it[nodeId].left, l, mid, u, v), get(it[nodeId].right, mid+1, r, u, v));
}


// When update:
    ++nVer;
    ver[nVer] = update(1, n, u, x, ver[nVer-1]);

// When query:
    res = get(ver[t], 1, n, u, v);
\end{lstlisting}

\subsection{Hash Map}
\begin{lstlisting}
//faster than unordered_map
struct hash_map {
    const static int SZ = 2e4 + 9;
    int nxt[SZ >> 3], val[SZ >> 3];
    int key[SZ >> 3];
    int h[SZ + 5], cnt;
    vector<int>vec;
    
    void clear(){
        for (int i : vec) h[i] = 0;
        for (int i = 1; i <= cnt; i++)
            val[i] = nxt[i] = 0, key[i] = 0;
 
        vec.clear();
        cnt = 0;
    }
    
    int hash(int u) {
        return u % SZ;
    }
    
    int &operator[](int u) {
        int x = hash(u);
 
        for (int i = h[x]; i; i = nxt[i])
            if (key[i] == u) return val[i];
 
        if (!h[x]) vec.push_back(x);
 
        ++cnt;
        key[cnt] = u;
        val[cnt] = 0;
        nxt[cnt] = h[x];
        h[x] = cnt;
        return val[cnt];
    }
    
    int qry(int u) {
        int x = hash(u);
        for (int i = h[x]; i; i = nxt[i])
            if (key[i] == u) return val[i];
 
        return 0;
    }
} hs;
\end{lstlisting}

\section {String}
\subsection{Trie 1}
\begin{lstlisting}
struct node{
	node *g[26];
	node(){
		rep(i, 26) g[i] = NULL;
	}
} *root = new node();

void Insert(string s){
	node *p = root;
	for (char t: s){
		if (p->g[t - 'a'] == NULL)
			p->g[t - 'a'] = new node();
		
		p = p->g[t - 'a'];	
	}
}
\end{lstlisting}

\subsection{Trie 2}
\begin{lstlisting}
int nNode = 0;
int g[N][26];

void Insert(string s){
    int p = 0;
    for (char t: s){
        if (!g[p][t - 'a']) g[p][t - 'a'] = ++nNode;
        p = g[p][t - 'a'];
    }
}
\end{lstlisting}

\subsection{Hash}
\begin{lstlisting}
ll getHashT(int i, int j) {
    return (hashT[j] - mul(hashT[i - 1], POW[j - i + 1]) + MOD) % MOD;
}

// Precalculate base^i
for (int i = 1; i <= lenT; i++)
    POW[i] = (POW[i - 1] * base) % MOD;

// Calculate hash value of T[1..i]
for (int i = 1; i <= lenT; i++)
    hashT[i] = (hashT[i - 1] * base + (T[i] - 'a' + 1)) % MOD;
\end{lstlisting}

\subsection{KMP}
\begin{lstlisting}
//prefix function: length of the longest prefix of the substring s[1..i] that is also a suffix of this same substring
int k = 0;
For(i, 2, n){ //1-indexed
    while (k && s[k + 1] != s[i]) k = kmp[k];
    kmp[i] = (s[k + 1] == s[i]) ? ++k : 0;
}
\end{lstlisting}

\subsection{Manacher}
\begin{lstlisting}
vector<int> manacher_odd(string s) {
    int n = s.size();
    s = "$" + s + "^";
    vector<int> p(n + 2);
    int l = 0, r = 1;
    for(int i = 1; i <= n; i++) {
        p[i] = min(r - i, p[l + (r - i)]);
        while(s[i - p[i]] == s[i + p[i]]) {
            p[i]++;
        }
        if(i + p[i] > r) {
            l = i - p[i], r = i + p[i];
        }
    }
    return vector<int>(begin(p) + 1, end(p) - 1);
}

vector<int> manacher(string s) {
    string t;
    for(auto c: s) {
        t += string("#") + c;
    }
    auto res = manacher_odd(t + "#");
    return vector<int>(begin(res) + 1, end(res) - 1);
}
\end{lstlisting}

\subsection{Aho - Corasick}
\begin{lstlisting}
namespace Trie{
	struct Node{
		int child[26], p = -1, cnt = 0;
		char pch;
		int link = -1, go[26];
		Node(int p = -1, char ch = '#'): p(p), pch(ch){
			fill(begin(child), end(child), -1);
        	fill(begin(go), end(go), -1);
		}
	};
	
	vector<Node> g(1);
	
	void add(string s){
		int v = 0;
		for (char t: s){
			int c = t - 'a';
			if (g[v].child[c] == -1){
				g[v].child[c] = g.size();
				g.emplace_back(v, t);
			}
			v = g[v].child[c];
		}
		g[v].cnt++;
	}
	
	int go(int v, char c);
	
	int get_link(int v){
		if (g[v].link == -1){
			if (!v || !g[v].p) g[v].link = 0;
			else g[v].link = go(get_link(g[v].p), g[v].pch);
		}
		return g[v].link;
	}
	
	int go(int v, char t){
		int c = t - 'a';
		if (g[v].go[c] == -1){
			if (g[v].child[c] != -1) g[v].go[c] = g[v].child[c];
			else g[v].go[c] = (v == 0) ? 0 : go(get_link(v), t);
		}
		return g[v].go[c];
	}
}
\end{lstlisting}
\subsection{Aho - Corasick (BFS)}
\begin{lstlisting}
struct trie{
	struct Node{
		Node *child[26], *link;
		int cnt = 0;
		Node(){
			cnt = 0;
			rep(i, 26) child[i] = NULL;
			link = NULL;
		}
	} *root = new Node();
	
	
	void add(string &s){
		Node* p = root;
		for (char &t: s){
			int c = t - 'a';
			if (p->child[c] == NULL) p->child[c] = new Node();;
			p = p->child[c];
		}
		p->cnt++;
	}
	
	void AhoCorasick(){
		root->link = root;
		queue<Node*> q; q.push(root);
		while (!q.empty()){
			Node* p = q.front(); q.pop();
			rep(i, 26) if (p->child[i]){
				Node* k = p->link;
				while (k != root && k->child[i] == NULL) k = k->link;
				
				if (k->child[i] && k != p) p->child[i]->link = k->child[i];
				else p->child[i]->link = root;
				
				p->child[i]->cnt += p->child[i]->link->cnt;
				q.push(p->child[i]);
			}
		}
	}
};
\end{lstlisting}

\section{SQRT Decomposition}
\subsection{MO}
\begin{lstlisting}
struct query{
	int l, r, id;
}

bool cmp(const query &a, const query &b){
    if(a.l / S != b.l / S) return a.l < b.l;  

    if((a.l / S) & 1)
        return a.r < b.r;
    else
        return a.r > b.r
}

long long res = 0;

void update(...);

///////////////////////////////
sort(q + 1, q + Q + 1, cmp);

int l = 1, r = 0;
for(int i = 1; i <= Q; i++){
    while(l < q[i].l) update(a[l++], ...);
    while(l > q[i].l) update(a[--l], ...);
    while(r < q[i].r) update(a[++r], ...);
    while(r > q[i].r) update(a[r--], ...);
    ans[q[i].id] = res;
}
\end{lstlisting}

\section{Graph}
\subsection{Joint and Bridge}
\begin{lstlisting}
void dfs(int u, int pre) {
    int child = 0; 
    num[u] = low[u] = ++timer;
    for (int v: g[u]) {
        if (v == pre) continue;
        if (!num[v]) {
            dfs(v, u);
            low[u] = min(low[u], low[v]);
            if (low[v] == num[v]) bridge++;
            child++;
            if (u == pre){
                if (child > 1) joint[u] = true;
            }
            else if (low[v] >= num[u]) joint[u] = true;
        }
        else low[u] = min(low[u], num[v]);
    }
}
\end{lstlisting}

\subsection{SCC}
\begin{lstlisting}
void dfs(int u) {
    num[u] = low[u] = ++timer;
    st.push(u);
    for (int v : g[u]) {
        if (!num[v]){
            dfs(v);
            low[u] = min(low[u], low[v]);
        }
        else low[u] = min(low[u], num[v]);
    }
    if (low[u] == num[u]) {
        scc++;
        int v;
        do {
            v = st.top();
            st.pop();
            num[v] = INF;
        }
        while (v != u);
    }
}
\end{lstlisting}

\subsection{Topology Sort 1}
\begin{lstlisting}
//u -> v
//++deg[v]
for (int u = 1; u <= n; ++u)
    if (!deg[u]) q.push(u);

while (!q.empty()) {
    int u = q.front();
    q.pop();
    topo.push_back(u);
    for (auto v : g[u]) {
        deg[v]--;
        if (!deg[v]) q.push(v);
    }
}
\end{lstlisting}

\subsection{Topology Sort 2}
\begin{lstlisting}
void dfs(int u) {
    visit[u] = 1;
    for (auto v : g[u]) {
        assert(visit[v] != 1); 
        //graph contains a cycle
        if (!visit[v]) dfs(v);
    }
    topo.push(u);
    visit[u] = 2;
}
\end{lstlisting}

\subsection{Max Flow}
\begin{lstlisting}
struct edge{
	int to, rev, flow, cap;
};

void add_edge(int u, int v, int cap){
	edge e1 = {v, sz(g[v]), 0, cap};
    edge e2 = {u, sz(g[u]), 0 , 0};
    g[u].pb(e1); g[v].pb(e2);
}

bool bfs(){
	memset(dist, 0x3f, sizeof dist);
	queue<int> q;
	q.push(source); dist[source] = 0;
	while (!q.empty()){
		int u = q.front(); q.pop();
		for (edge e: g[u]){
			int v = e.to, flow = e.flow, cap = e.cap;
			if (flow < cap && minimize(dist[v], dist[u] + 1))
				q.push(v);
		}
	}
	return dist[sink] < INF;
}

int dfs(int u, int mn){
	if (u == sink) return mn;
	for (int &i = lazy[u]; i < sz(g[u]); ++i){
		auto &[v, rev, flow, cap] = g[u][i];
		if (dist[v] == dist[u] + 1 && flow < cap){
			int cur = dfs(v, min(mn, cap - flow));
			if (cur > 0){
				flow += cur;
				g[v][rev].flow -= cur;
				return cur;
			} 
		}
	}
	return 0;
}

int main(){
    //...
    int res = 0;
    while (bfs()){
        memset(lazy, 0, sizeof lazy);
        while (int del = dfs(source, INF)) 
            res += del;
    }
        
    cout << res;
    return 0;
}
\end{lstlisting}

\subsection{Bipartite Matching}
\begin{lstlisting}
bool dfs(int u){
	if (seen[u]) return 0;
	seen[u] = 1;
	
	for (int v: g[u]) 
		if (!mt[v] || dfs(mt[v]))
			return mt[v] = u, 1;
	
	return 0;
}

//memset(mt, 0, sizeof mt);
//For(i, 1, n){
    //memset(seen, 0, sizeof seen);
    //dfs(i);
//}
\end{lstlisting}

\subsection{HLD}
\begin{lstlisting}
void dfs(int u){
	sz[u] = 1;
	for (int v: g[u]) if (v != par[u]){
		par[v] = u;
		dfs(v);
		sz[u] += sz[v];
	}
}
 
void hld(int u){
	if (!Head[nChain]) Head[nChain] = u;
	idChain[u] = nChain;
	
	pos[u] = ++timer;
	node[timer] = u;
	
	int bigC = 0;
	for (int v: g[u]) if (v != par[u])
		if (!bigC || sz[v] > sz[bigC])
			bigC = v;
	
	if (bigC) hld(bigC);
	for (int v: g[u]) if (v != par[u] && v != bigC){
		++nChain;
		hld(v);;
	} 
}

//LCA
int LCA(int u, int v){
	while (idChain[u] != idChain[v]){
		if (idChain[u] > idChain[v]) 
			u = par[Head[idChain[u]]];
		else 
			v = par[Head[idChain[v]]];
	}
	
	if (h[u] < h[v]) return u;
	return v;
}

int get(int u, int v){
	int res = 0;
	while (idChain[u] != idChain[v]){
		if (idChain[u] > idChain[v]){
		    maximize(res, ST.get(pos[Head[idChain[u]]], pos[u]));
		    u = par[Head[idChain[u]]];
		}
		else{
		    maximize(res, ST.get(pos[Head[idChain[v]]], pos[v]));
	    	v = par[Head[idChain[v]]];
		}
	}

	if (pos[u] < pos[v]) 
		maximize(res, ST.get(pos[u], pos[v]));
	else
		maximize(res, ST.get(pos[v], pos[u]));
	
	return res;
}
\end{lstlisting}

\subsection{DSU on tree}
\begin{lstlisting}
void dfs(int u, int prev = -1){
	in[u] = ++timer; node[timer] = u;
	for (int v: g[u]) if (v != prev)
		dfs(v, u);
	out[u] = timer;
}
 
#define sz(u) out[u] - in[u]
 
void calc(int u, int prev = -1){
	int bigC = 0;
	for (int v: g[u]) if (v != prev)
		if (sz(v) > sz(bigC))
			bigC = v;
	
	for (int v: g[u]) if (v != prev && v != bigC){
        calc(v, u);
        //reset(v)...
    }
		
	if (bigC) calc(bigC, u);
	
	for (int v: g[u]) if (v != prev && v != bigC){
		For(t, in[v], out[v]){
			int x = node[t];
			//...
		}
	}
}
\end{lstlisting}

\subsection{Centroid Decomposition}
\begin{lstlisting}
int size(int u, int prev){
	sz[u] = 1;
	for (int v: g[u]) if (!del[v] && v != prev)
		sz[u] += size(v, u);
	return sz[u];
}

int centroid(int u, int prev){
	for (int v: g[u]) if (!del[v] && v != prev)
		if (sz[v] > n/2)
			return centroid(v, u);
	return u;
}

void dfs(int u, int prev){
    in[u] = ++timer; node[timer] = u;
    for (int v: g[u]) if (!del[v] && v != prev){
        dfs(v, u);
        //...
    }
    out[u] = timer;
}

void calc(int u){
	n = size(u, 0);
	u = centroid(u, 0);
	
	timer = 0;
	dfs(u, 0);
	
    for (int v: g[u]) if (!del[v]){
        //subtree v...
    }
    //reset
	del[u] = 1;
	for (auto [v, c]: g[u]) if (!del[v])
		calc(v);
}
\end{lstlisting}

\subsection{Centroid Tree (CT)}
Centroid Tree properties:
\begin{itemize}
    \item Centroid tree height $ \leq \log(n) $
    \item $LCA(u, v)$ in CT lies on the path from $u$ to $v$ in the original tree
\end{itemize}


\begin{lstlisting}
int size(int u, int prev){
	sz[u] = 1;
	for (int v: g[u]) if (v != prev && !del[v]){
		sz[u] += size(v, u);
	}
    return sz[u];
}
 
int centroid(int u, int prev, int m){
	for (int v: g[u]) if (v != prev && !del[v])
		if (sz[v] > m/2) 
			return centroid(v, u, m);
	return u;
}
 
int cd(int u){
	int m = size(u);
	u = centroid(u, 0, m);
	del[u] = 1;
	for (int v: g[u]) if (!del[v]){
		v = cd(v);
		par[v] = u;
	}
	return u;
}

//example problems:
void solve(){
    dfs(1, 0); init(); //to calculate the dist(u, v) from the original tree
	cd(1);
	
	memset(d, 0x3f, sizeof d);
	
	c[1] = 1; //color
	int pp = 1;
	while (pp){
		minimize(d[pp], dist(pp, 1));
		pp = par[pp];
	}
	
	while (q--){
		int t; cin >> t;
		if (t == 1){
			int u; cin >> u;
			c[u] = 1;
			
			int p = u;
			while (p){
				minimize(d[p], dist(p, u));
				p = par[p];
			}
		}
		else{
			int u; cin >> u;
			if (c[u]) {
                cout << 0 << endl; continue;
            }
			
			int p = u, res = INF;
			while(p){
				minimize(res, dist(u, p) + d[p]);
				p = par[p];
			}
			
			cout << res << endl;
		}
	}
}
\end{lstlisting}

\subsection{Virtual Tree}
\begin{lstlisting}
void dfs(int u){
	in[u] = ++timer;
	for (int v: g[u]) if (v != up[u][0]){
		up[v][0] = u;
		For(j, 1, 17) up[v][j] = up[up[v][j - 1]][j - 1];
		dfs(v);
	}
	out[u] = timer;
}
 
bool is_anc(int u, int v){
	if (!u) return 1;
	return in[u] <= in[v] && in[v] <= out[u];
}

//short LCA
int lca(int u, int v){
	if (is_anc(u, v)) return u;
	ForD(j, 17, 0){
        if (!is_anc(up[u][j], v)){
            u = up[u][j];
        }
    }
	return up[u][0];
}
 
bool cmp(int u, int v){
	return in[u] < in[v];
}

void query(){
	cin >> k;
	For(i, 1, k) cin >> a[i], sz[a[i]] = 1;
	
	sort(a + 1, a + k + 1, cmp);
	For(i, 1, k - 1) a[i + k] = lca(a[i], a[i + 1]);
	
	sort(a + 1, a + k + k, cmp);
	k = unique(a + 1, a + k + k) - a - 1;
	
	stack<int> st; st.push(a[1]);
	For(i, 2, k){
		while (!is_anc(st.top(), a[i])) st.pop();
		g[st.top()].pb(a[i]);
		st.push(a[i]);
	}
	
	res = 0; calc(a[1]);
	cout << res << endl;
	
	For(i, 1, k) sz[a[i]] = 0, g[a[i]].clear();
}

void solve(){
	//...	
	dfs(1);
	For(i, 1, n) g[i].clear();
	while (q--) query();
}
\end{lstlisting}

\section {DP}
\subsection{Digit DP}
\begin{lstlisting}
int f(int id, bool sml, ...){
    if (id < 0) return ...;
    if (!sml && dp[id][...] != -1) return dp[id][...];

    int lim = sml ? a[id] : 9;
    int res = ...;
    For(c, 0, lim){
        update(res, f(id - 1, sml && c == lim, ...));
    }
    if (!sml) dp[id][...] = res;
    return res;
}

int get(int x){
    int n = 0;
    while (x){
        a[n++] = x % 10;
        x /= 10;
    }
    return f(n - 1, 1, ...);
}
\end{lstlisting}

\subsection{SOS DP}
\begin{lstlisting}
for (int k = 0; k < n; k++)
    for (int mask = 0; mask < (1 << n); mask++)
        if (mask & (1 << k)) 
            dp[mask] += dp[mask ^ (1 << k)];
\end{lstlisting}

\subsection{DP DNC}

Solving problems that have the form:
\[
dp(i, j) =
\min\limits_{0 \le k \le j} \Big( dp(i - 1, k - 1) + C(k, j) \Big)
\]
The cost function has to satisfie the quadrangle inequality:
$C(a, c) + C(b, d) \leq C(a, d) + C(b, c)$ for all $a \leq b \leq c \leq d$.

Example:

$C(j, i) = f(i - j)$ where $f$ is convex.

$C(j, i) = (i - j)$

$C(j, i) = (i - j)^2$
\begin{lstlisting}
int m, n;
vector<int> dp_before(n + 1), dp_cur(n + 1);

// cost function
int C(int l, int r);

// calculate dp_cur[l], ..., dp_cur[r]
void compute(int l, int r, int optl, int optr) {
    if (l > r) return;

    int mid             = (l + r) >> 1;
    pair<int, int> best = {INT_MAX, -1};
    //calculate dp_cur[mid] & opt[i][mid] depend on dp_before and cost func
    for (int k = optl; k <= min(mid, optr); ++k) {
        minimize(best, {dp_before[k] + C(k, mid), k})
    }
    dp_cur[mid] = best.first;
    int opt     = best.second;

    compute(l, mid - 1, optl, opt);
    compute(mid + 1, r, opt, optr);
}

int solve() {
    for (int i = 0; i <= n; ++i)
        dp_before[i] = C(0, i);

    for (int i = 1; i < m; ++i) {
        compute(0, n, 0, n);
        dp_before = dp_cur;
    }

    return dp_before[n];
}
\end{lstlisting}

\subsection{Convex Hull Trick}
Adding lines $y = kx + m$ and querying minimum values at integer $x$.
\begin{lstlisting}
struct Line {
    int k, m;
    mutable int p;
    
    int eval(int x){
    	return k * x + m;
    }
    
    bool operator < (const Line& l) const {
        return k < l.k;
    }
    bool operator < (const int &x) const {
        return p < x;
    }
};
 
struct ConvexHull : multiset<Line, less<>> {
    int div(int a, int b) {
        return a / b - ((a ^ b) < 0 && a % b);
    }

    bool bad(iterator x, iterator y) {
        if(y == end()) { 
            x->p = LINF;
            return 0; 
        }

        if(x->k == y->k) x->p = x->m > y->m ? LINF : -LINF;
        else x->p = div(y->m - x->m, x->k - y->k);

        return x->p >= y->p;
    }

    void add(int k, int m) {
        auto z = insert({k, m, 0}), y = z++, x = y;
        while (bad(y, z)) z = erase(z);

        if(x != begin() && bad(--x, y)) bad(x, y = erase(y));
        while((y = x) != begin() && (--x)->p >= y->p) bad(x, erase(y));
    }

    int query(int x) {
        assert(!empty());
        Line l = *lower_bound(x);
        return l.eval(x);
    }
} CH;

//If you want maximum, just flip signs: 
CH.add(-k, -m);
int res = -CH.query(x);
\end{lstlisting}

\subsection{1D1D Optimization}
Solving problems that have the form: 
\[
dp(i) = \min_{0 \leq j < i} \Big( dp(j) + C(j, i) \Big),
\]
\begin{lstlisting}
struct item {
  int l, r, p;
};

long long w(int j, int i) {
    //cost function
}

void solve() {
  deque<item> dq;
  dq.push_back({1, n, 0});
  for (int i = 1; i <= n; ++i) {
    f[i]=f[dq.front().p]+w(dq.front().p,i);
    ++dq.front().l;

    if (dq.front().l > dq.front().r) {
      dq.pop_front();
    }

    while (!dq.empty()) {
      auto [l, r, p] = dq.back();
      if (f[i] + w(i, l) < f[p] + w(p, l)) {
        dq.pop_back();
      }
      else break;
    }

    if (dq.empty()) {
      dq.push_back({i + 1, n, i});
      // h[i+1]=h[i+2]=...=h[n]=i
    }
    else {
      auto& [l, r, p] = dq.back();
      int low = l, high = r;
      int pos = r + 1, mid;
      while (low <= high) {
        mid = (low + high) / 2;
        if (f[i] + w(i, mid) < f[p] + w(p, mid)) {
          pos = mid, high = mid - 1;
        }
        else {
          low = mid + 1;
        }
      }

      r = pos - 1;
      if (pos <= n) {
        dq.push_back({pos, n, i});
        // h[pos]=h[pos+1]=...=h[n]=i
      }
    }
  }
}
\end{lstlisting}


\subsection{Knapsack on Tree}
Problem: Given a tree $T$ with $N$ vertices rooted at vertex $1$ 
($1 \leq N \leq 5000$). 
The $i$-th vertex has a value $C_i$ and a constraint $K_i$ 
($|C_i| \leq 10^9$, $1 \leq K_i \leq N$). 
Choose a subset of vertices such that, in the subtree of every vertex $i$, 
there are at most $K_i$ chosen vertices, and the total sum of the chosen 
vertices' values is maximized.

\begin{lstlisting}
void calc(int V){
    int n = child[V].size();

    for(int v_i: child[V]) {
        calc(v_i);
    }

    for(int i = 0; i <= n; i++)fill(fV[i], fV[i] + N + 1, -INF);    
    fV[0][0] = 0;                                                   

    for(int i = 1; i <= n; i++){
        int v_i = child[V][i - 1];
        for(int a = 0; a <= sz[V]; a++){        
            for(int b = 0; b <= sz[v_i]; b++){
                fV[i][a+b] = max(fV[i][a+b], fV[i-1][a] + dp[v_i][b]);  
        }
        sz[V] += sz[v_i];
    }

    for(int k = 0; k <= N; k++){
        if(k > K[V])dp[V][k] = -INF;                                
        else {                                                      
            if(k > 0)dp[V][k] = max(fV[n][k], fV[n][k-1] + C[V]);   
            else dp[V][k] = fV[n][k];                               
        }
    }
    sz[V]++;
}

long long solve() {
    calc(1);
    return *max_element(dp[1], dp[1] + N + 1);
}
\end{lstlisting}
\section{Math}
\subsection{Euler's totient function}
\begin{lstlisting}
int phi(int n) {
    int res = n;
    for (int i = 2; i * i <= n; i++) {
        if (n % i == 0) {
            while (n % i == 0) n /= i;
            res -= res / i;
        }
    }
    if (n > 1) res -= res / n;
    return res;
}
\end{lstlisting}

\subsection{Euler's totient function from $1$ to $N$}
\begin{lstlisting}
void preCompute(int n) {
    iota(phi, phi + N, 0); //phi[i] = i
    for (int i = 2; i <= n; i++) {
        if (phi[i] == i) {
            for (int j = i; j <= n; j += i)
                phi[j] -= phi[j] / i;
        }
    }
}
\end{lstlisting}

\subsection{Modular Inverse}
\begin{lstlisting}
//if MOD is a prime number then phi(MOD)= MOD - 1
int inv(int x, int MOD){
    return Pow(x, phi(MOD) - 1);
}
\end{lstlisting}

\subsection{Extended Euclidean Algorithm}
\begin{lstlisting}
//computing gcd(a, b) and finding (x, y) that
//ax + by = gcd(a, b)

//recursive version
int gcd(int a, int b, int& x, int& y) {
    if (b == 0) {
        x = 1; y = 0;
        return a;
    }
    int x1, y1;
    int d = gcd(b, a % b, x1, y1);
    x = y1;
    y = x1 - y1 * (a / b);
    return d;
}

//iterative version
int gcd(int a, int b, int& x, int& y) {
    x = 1, y = 0;
    int x1 = 0, y1 = 1, a1 = a, b1 = b;
    while (b1) {
        int q = a1 / b1;
        tie(x, x1) = make_tuple(x1, x - q * x1);
        tie(y, y1) = make_tuple(y1, y - q * y1);
        tie(a1, b1) = make_tuple(b1, a1 - q * b1);
    }
    return a1;
}
\end{lstlisting}

\subsection{Diophantine}
\begin{lstlisting}
bool find_any_solution(int a, int b, int c, int &x0, int &y0, int &g) {
    g = gcd(abs(a), abs(b), x0, y0);
    if (c % g) return false;

    x0 *= c / g; y0 *= c / g;
    if (a < 0) x0 = -x0;
    if (b < 0) y0 = -y0;
    return true;
}

//all the solutions have the form:
//x = x0 + k * b/g
//y = y0 - k * b/g


//IN A GIVEN INTERVAL:
void shift(int & x, int & y, int a, int b, int cnt) {
    x += cnt * b;
    y -= cnt * a;
}

int find_all_solutions(int a, int b, int c, int minx, int maxx, int miny, int maxy) {
    int x, y, g;
    if (!find_any_solution(a, b, c, x, y, g)) return 0;
    a /= g; b /= g;

    int sign_a = a > 0 ? +1 : -1;
    int sign_b = b > 0 ? +1 : -1;

    shift(x, y, a, b, (minx - x) / b);
    if (x < minx) shift(x, y, a, b, sign_b);
    if (x > maxx) return 0;
    int lx1 = x;

    shift(x, y, a, b, (maxx - x) / b);
    if (x > maxx) shift(x, y, a, b, -sign_b);
    int rx1 = x;

    shift(x, y, a, b, -(miny - y) / a);
    if (y < miny) shift(x, y, a, b, -sign_a);
    if (y > maxy) return 0;
    int lx2 = x;

    shift(x, y, a, b, -(maxy - y) / a);
    if (y > maxy) shift(x, y, a, b, sign_a);
    int rx2 = x;

    if (lx2 > rx2) swap(lx2, rx2);
    int lx = max(lx1, lx2);
    int rx = min(rx1, rx2);

    if (lx > rx) return 0;
    return (rx - lx) / abs(b) + 1;
}
\end{lstlisting}

\subsection{Chinese Remainder Theorem}
\begin{lstlisting}
// Combine two congruences: 
// x = a1 (mod m1), x = a2 (mod m2)
// Returns (x, lcm) or (-1,-1) if no solution
pair<ll, ll> crt2(ll a1, ll m1, ll a2, ll m2) {
    int x, y;
    ll g = gcd(m1, m2, x, y);

    if ((a2 - a1) % g != 0) {
        return {-1, -1}; // no solution
    }

    ll lcm = m1 / g * m2;

    ll k = (a2 - a1) / g;
    ll mult = (1LL * x * k) % (m2 / g);

    ll ans = (a1 + m1 * mult) % lcm;
    if (ans < 0) ans += lcm;

    return {ans, lcm};
}

//solve a system of congruences:
//x = a1 (mod m1)
//x = a2 (mod m2)
//...
//x = ak (mod mk)
pair<ll, ll> crt(vector<ll> a, vector<ll> m) {
    pair<ll,ll> res = {a[0], m[0]};
    for (int i = 1; i < sz(a); i++) {
        res = crt2(res.first, res.second, a[i], m[i]);
        if (res.first == -1) return {-1,-1};
    }
    return res;
}

//x = sol.first (mod sol.second)
\end{lstlisting}

\subsection{Rabin-Miller primality test}
\begin{lstlisting}
bool test(ll a, ll n, ll k, ll m){
    ll mod = Pow(a, m, n);
    if (mod == 1 || mod == n - 1) return 1;
    for (int l = 1; l < k; ++l){
        mod = (mod * mod) % n;
        if (mod == n - 1) return 1;
    }
    return 0;
}

//check if n is a prime number
bool RabinMiller(ll n){
    if (n == 2 || n == 3 || n == 5 || n == 7) return 1;
    if (n < 11) return 0;

    ll k = 0, m = n - 1;
    while (!(m & 1)){
        m >>= 1;
        k++;
    }

    const static int repeatTime = 3;
    for (int i = 0; i < repeatTime; ++i){
        ll a = rand() % (n - 3) + 2;
        if (!test(a, n, k, m)) return 0;
    }
    return 1;
}
\end{lstlisting}

\subsection{Combination}
\begin{lstlisting}
//recursive version:
void preCompute(){
    for (int i = 0; i <= n; i++){
        C[i][0] = 1;
        for (int k = 1; k <= i; k++){
            C[i][k] = C[i - 1][k - 1] + C[i - 1][k];
        }
    }
}

//"you know what it is" version:
int C(int n, int k){
    if (n < k || k < 0) return 0;
    return mul(fact[n], mul(ifact[n - k], ifact[k])) 
}
\end{lstlisting}


\section{Geometry}
\

\end{multicols*}

\end{document}
